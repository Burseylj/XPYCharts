\documentclass{article}

\usepackage{booktabs}
\usepackage{tabularx}

\title{SE 3XA3: Development Plan\\Title of Project}

\author{Team \#, Team Name
		\\ Student 1 name and macid
		\\ Student 2 name and macid
		\\ Student 3 name and macid
}

\date{}

%% Comments

\usepackage{color}

\newif\ifcomments\commentstrue

\ifcomments
\newcommand{\authornote}[3]{\textcolor{#1}{[#3 ---#2]}}
\newcommand{\todo}[1]{\textcolor{red}{[TODO: #1]}}
\else
\newcommand{\authornote}[3]{}
\newcommand{\todo}[1]{}
\fi

\newcommand{\wss}[1]{\authornote{blue}{SS}{#1}}
\newcommand{\ds}[1]{\authornote{red}{DS}{#1}}
\newcommand{\mj}[1]{\authornote{red}{MSN}{#1}}
\newcommand{\cm}[1]{\authornote{red}{CM}{#1}}
\newcommand{\mh}[1]{\authornote{red}{MH}{#1}}

% team members should be added for each team, like the following
% all comments left by the TAs or the instructor should be addressed
% by a corresponding comment from the Team

\newcommand{\tm}[1]{\authornote{magenta}{Team}{#1}}


\begin{document}

\begin{table}[hp]
\caption{Revision History} \label{TblRevisionHistory}
\begin{tabularx}{\textwidth}{llX}
\toprule
\textbf{Date} & \textbf{Developer(s)} & \textbf{Change}\\
\midrule
Date1 & Name(s) & Description of changes\\
Date2 & Name(s) & Description of changes\\
... & ... & ...\\
\bottomrule
\end{tabularx}
\end{table}

\newpage

\maketitle
%Hatim
Put your introductory blurb here.

\section{Team Meeting Plan} %Louis
Our team will hold regular meetings on Tuesdays and Thursdays, for about two hours. We will meet either in a free room in MDCL or in the Thode Library.

Meetings will begin with a discussion of the tasks completed for this meeting. If a task has not been completed, a plan will be made to complete it, and we will discuss the reasons this task was not completed.

  Meetings will be chaired by the person most knowledgeable about the meeting's focus. Team members should come to the meeting with their tasks completed, aware of the current state of the project, and aware of the next steps planned. They should also prepare agenda items that they feel important to discuss. 

  The Scribe will take minutes and draft agendas from the items brought by team members and upcoming milestones. Each agenda item should be phrased as a question, and should be unbiased, e.g.\ \textit{What Language Should We Use} is better practice than \textit{Should We Use C++?}.

  The meeting should follow the structure of the agenda. Important decisions will ideally be decided by consensus. If a discussion stagnates and moves alloted time without the group reaching a consensus, a decision will be made by vote.
  
  Meetings will end with the scribe writing a brief summary of the decisions made and the tasks to be completed. We will also briefly reflect on the effectiveness of the meeting, and the things that can be done to make meetings more productive.
\section{Team Communication Plan} %Sarthak

\section{Team Member Roles} %Sarthak

\section{Git Workflow Plan}
%ask ta

\section{Proof of Concept Demonstration Plan} %Hatim

\section{Technology} %Louis
This project will be developed in the Python programming language. To aid in project management, we will use the free version of the PyCharm IDE.  For automated testing, we will use the ImageMagick image comparison tool and the test files provided in the JCharts project.

\section{Coding Style} %Louis
We will adhere to the PEP 8 for Python style. Thorough and meaningful commenting is expected in all code.

\section{Project Schedule}

%Will have pointer to Gantt chart

\section{Project Review} 
%igonore for 1st iteration

\end{document}