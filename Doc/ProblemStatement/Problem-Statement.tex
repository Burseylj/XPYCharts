% !TEX TS-program = pdflatex
% !TEX encoding = UTF-8 Unicode

% This is a simple template for a LaTeX document using the "article" class.
% See "book", "report", "letter" for other types of document.

\documentclass[11pt]{article} % use larger type; default would be 10pt

\usepackage[utf8]{inputenc} % set input encoding (not needed with XeLaTeX)

%%% Examples of Article customizations
% These packages are optional, depending whether you want the features they provide.
% See the LaTeX Companion or other references for full information.

%%% PAGE DIMENSIONS
\usepackage{geometry} % to change the page dimensions
\geometry{a4paper} % or letterpaper (US) or a5paper or....
% \geometry{margin=2in} % for example, change the margins to 2 inches all round
% \geometry{landscape} % set up the page for landscape
%   read geometry.pdf for detailed page layout information

\usepackage{graphicx} % support the \includegraphics command and options

% \usepackage[parfill]{parskip} % Activate to begin paragraphs with an empty line rather than an indent

%%% PACKAGES
\usepackage{booktabs} % for much better looking tables
\usepackage{array} % for better arrays (eg matrices) in maths
\usepackage{paralist} % very flexible & customisable lists (eg. enumerate/itemize, etc.)
\usepackage{verbatim} % adds environment for commenting out blocks of text & for better verbatim
\usepackage{subfig} % make it possible to include more than one captioned figure/table in a single float
% These packages are all incorporated in the memoir class to one degree or another...

%%% HEADERS & FOOTERS
\usepackage{fancyhdr} % This should be set AFTER setting up the page geometry
\pagestyle{fancy} % options: empty , plain , fancy
\renewcommand{\headrulewidth}{0pt} % customise the layout...
\lhead{}\chead{}\rhead{}
\lfoot{}\cfoot{\thepage}\rfoot{}

%%% SECTION TITLE APPEARANCE
\usepackage{sectsty}
\allsectionsfont{\sffamily\mdseries\upshape} % (See the fntguide.pdf for font help)
% (This matches ConTeXt defaults)

%%% ToC (table of contents) APPEARANCE
\usepackage[nottoc,notlof,notlot]{tocbibind} % Put the bibliography in the ToC
\usepackage[titles,subfigure]{tocloft} % Alter the style of the Table of Contents
\renewcommand{\cftsecfont}{\rmfamily\mdseries\upshape}
\renewcommand{\cftsecpagefont}{\rmfamily\mdseries\upshape} % No bold!

\usepackage{color}

%%% END Article customizations

%%% The "real" document content comes below...

\title{SFWR 3XA3: Problem Statement}
\author{Group 4: Sarthak Desai, Hatim Rehman, Louis Bursey}
\date{December 7, 2016} % Activate to display a given date or no date (if empty),
         % otherwise the current date is printed 

\begin{document}
\maketitle

\section*{Problem}

Developers face many challenges when they need to produce graphs their applications. Writing code that generates graphs reliably and correctly is time consuming, and often a problem that developers do not want to solve. A lightweight, easy to integrate, and reusable library that developers can use in any application would solve this problem. An existing solution, JCharts, will be reproduced.



\section*{Why is it important}

The use of graphing technology is very diverse, whether it is related to statistical analysis and mapping trends or used in an educational environment such as a classroom. Considering how widely useful such graphing tools can be and how their use can lead to better understanding of data, our group will be reproducing a graphing library.

\textcolor{red}{Current solutions to this problem are bulky and complicated. Often the time saved by using a library is reduced, because programmers have to spend time learning a new library. Focus on usability will save the user time, and better solve the problem.}

\section*{Context}

The stakeholders of this project include any programmers or developers who need to generate graphs in their application, any end user viewing the graphs generated by the library, and us, the development team. \textcolor{red}{It is important to programmers to produce accurate, visually appealing graphs quickly and easily. This allows them to focus on their programs, and abstract away the duty of creating graphs. Without accuracy and visual appeal, the programmer will need to implement their own, better solution. Without usability, the programmer will spend time trying to get the library working that would be better spent on their program. End users viewing the graphs will appreciate visual appeal, as it will make the graphs easier to interpret.}

\end{document}
