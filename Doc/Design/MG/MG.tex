\documentclass[12pt, titlepage]{article}

\usepackage{fullpage}
\usepackage[round]{natbib}
\usepackage{multirow}
\usepackage{booktabs}
\usepackage{tabularx}
\usepackage{graphicx}
\usepackage{float}
\usepackage{hyperref}
\hypersetup{
    colorlinks,
    citecolor=black,
    filecolor=black,
    linkcolor=red,
    urlcolor=blue
}
\usepackage[round]{natbib}

\newcounter{acnum}
\newcommand{\actheacnum}{AC\theacnum}
\newcommand{\acref}[1]{AC\ref{#1}}

\newcounter{ucnum}
\newcommand{\uctheucnum}{UC\theucnum}
\newcommand{\uref}[1]{UC\ref{#1}}

\newcounter{mnum}
\newcommand{\mthemnum}{M\themnum}
\newcommand{\mref}[1]{M\ref{#1}}

\title{SE 3XA3: Software Requirements Specification\\Title of Project}

\author{Team 4, xPy
		\\ Hatim Rehman (rehmah3)
		\\ Louis Bursey (burseylj)
		\\ Sarthak Desai (desaisa3)
}

\date{\today}

%% Comments

\usepackage{color}

\newif\ifcomments\commentstrue

\ifcomments
\newcommand{\authornote}[3]{\textcolor{#1}{[#3 ---#2]}}
\newcommand{\todo}[1]{\textcolor{red}{[TODO: #1]}}
\else
\newcommand{\authornote}[3]{}
\newcommand{\todo}[1]{}
\fi

\newcommand{\wss}[1]{\authornote{blue}{SS}{#1}}
\newcommand{\ds}[1]{\authornote{red}{DS}{#1}}
\newcommand{\mj}[1]{\authornote{red}{MSN}{#1}}
\newcommand{\cm}[1]{\authornote{red}{CM}{#1}}
\newcommand{\mh}[1]{\authornote{red}{MH}{#1}}

% team members should be added for each team, like the following
% all comments left by the TAs or the instructor should be addressed
% by a corresponding comment from the Team

\newcommand{\tm}[1]{\authornote{magenta}{Team}{#1}}


\begin{document}

\maketitle

\pagenumbering{roman}
\tableofcontents
\listoftables
\listoffigures

\begin{table}[bp]
\caption{\bf Revision History}
\begin{tabularx}{\textwidth}{p{3cm}p{2cm}X}
\toprule {\bf Date} & {\bf Version} & {\bf Notes}\\
\midrule
Date 1 & 1.0 & Notes\\
Date 2 & 1.1 & Notes\\
\bottomrule
\end{tabularx}
\end{table}

\newpage

\pagenumbering{arabic}

\section{Introduction}
\subsection{Overview}
XPYCharts is a python based graphing library that is a recreation of the Jcharts graphing library available for Java developers. XPYCharts supports the production of scatter plots, line graphs and pie charts that can be used for statistical analysis and in a learning environment for data visualization.
\subsection{Context}
This document will serve as the Module Guide (MG) for the SRS document of XPYCharts. The MG shows a decomposition of the modules of the software system and how these modules accomplish the functional/non-functional requirements previously stated within the SRS. The MG will also show how the software adheres to the principals of information hiding and data independence in order to create a modular structure. The MG document is followed by the Module Interface Specification (MIS) that describes from an external viewpoint the intended behavior of the module’s functions. 
\subsection{Design Principals}
The software system of XPYCharts is built with consideration to several design principals including Modular decomposition, data independence, encapsulation and information hiding. In addition the decomposed modules of the system interact with each other through a structure that ensures low coupling and high cohesion so that modules stay as independent of each other as possible. This type of design will allow for easier software maintenance and modification in the future releases of the product. Similarly information hiding allows secrecy to exist over the module implementation making it easier to use the modules at a higher level without having to worry about the way they were implemented. This is especially important to XPYCharts as it is a library and therefore developers using the library should not have to know how the library modules are implemented to be able to use them.
\subsection{Document Structure}
The document is divided into 7 major sections including the Introduction above. Listed below is each of the major sections (excluding Introduction) with a brief description of the information that they contain:
\begin{itemize}
\item Anticipated and Unlikely Changes (Section 2): This section outlines the changes that are likely and unlikely to happen. This section is split into 2 subsections covering all likely and unlikely changes.
\item Module Hierarchy (Section 3): This section provides an outline of the module design by detailing the module hierarchy and providing a table describing the secrets of the modules.
\item Connection between Requirements and Design (Section 4): This section maps out the modules that implement each of the requirements as presented in the SRS document.
\item Module Decomposition (Section 5): This section decomposes the modules by their secrets and states the design decisions that are satisfied by these secrets. The section focuses on the purpose of the module’s existence rather than how it is implemented.
\item Traceability Matrix (Section 6): This section provides two tractability matrices that allow tracking of which requirements were satisfied by which modules and which modules will take care of which anticipated changes.
\item Uses Hierarchy Between Modules (Section 7): This section shows the uses relationships between the modules of the system.
\end{itemize}


\section{Anticipated and Unlikely Changes} \label{SecChange}

This section lists possible changes to the system. According to the likeliness
of the change, the possible changes are classified into two
categories. Anticipated changes are listed in Section \ref{SecAchange}, and
unlikely changes are listed in Section \ref{SecUchange}.

\subsection{Anticipated Changes} \label{SecAchange}

Anticipated changes are the source of the information that is to be hidden
inside the modules. Ideally, changing one of the anticipated changes will only
require changing the one module that hides the associated decision. The approach
adapted here is called design for
change.

\begin{description}
\item[\refstepcounter{acnum} \actheacnum \label{acHardware}:] The specific
  hardware on which the software is running.
\item[\refstepcounter{acnum} \actheacnum \label{acInput}:] The format of the
  initial input data.
\item [\refstepcounter{acnum} \actheacnum \label{acGraphType}:] Additional graph types on Cartesian grids.
\end{description}

\subsection{Unlikely Changes} \label{SecUchange}

The following are the changes in the system that are unlikely to happen by the software release and in the future updates:

\begin{description}
\item[\refstepcounter{ucnum} \uctheucnum \label{ucIO}:] The purpose (to create a graphing library accessible to python developers) and scope (i.e. reduction of scope. Project scope is open to expansion) of the project.
\item[\refstepcounter{ucnum} \uctheucnum \label{ucInput}:] The parsing mechanism for the data that is inputted by the user and the data structure developed from it (the library converts the input data into a dictionary of x-y co-ordinate pairs).
\item[\refstepcounter{ucnum} \uctheucnum \label{ucInput}:]The data input methods in terms how the library expects user input (i.e. the current implementation allows developers to enter all data points as co-ordinate pairs inside of a list).
\item[\refstepcounter{ucnum} \uctheucnum \label{ucInput}:]The platform for the library (i.e. the library will run on Python 2).
\end{description}

\section{Module Hierarchy} \label{SecMH}

This section provides an overview of the module design. Modules are summarized
in a hierarchy decomposed by secrets in Table \ref{TblMH}. The modules listed
below, which are leaves in the hierarchy tree, are the modules that will
actually be implemented.

\begin{description}
\item [\refstepcounter{mnum} \mthemnum \label{mHH}:] Hardware-Hiding Module \\ 
\setlength\parindent{24pt} M1.1 \label{mOW}: Output Window Module 
\item [\refstepcounter{mnum} \mthemnum \label{mBH}:] Behaviour-Hiding Module \\
\setlength\parindent{24pt} M2.1 \label{mER}: Exception Raising Module \\
\setlength\parindent{24pt} M2.2 \label{mOF}: Output Format Module \\
\setlength\parindent{24pt} M2.3 \label{mPE}: Polynomial Equations Module
\item [\refstepcounter{mnum} \mthemnum \label{mSD}:] Software Decision Module \\
\setlength\parindent{24pt} M3.1 \label{mDS}: Data Structure Module \\
\setlength\parindent{24pt} M3.2 \label{mPS}: Polynomial Solver Module \\
\setlength\parindent{24pt} M3.3 \label{mP}: Plotting Module
\end{description}


\begin{table}[h!]
\centering
\begin{tabular}{p{0.3\textwidth} p{0.6\textwidth}}
\toprule
\textbf{Level 1} & \textbf{Level 2}\\
\midrule

{Hardware-Hiding Module} 
& Output Window Module \\
\midrule

\multirow{4}{0.3\textwidth}{Behaviour-Hiding Module} 
& Exception Raising Module\\
& Output Format Module\\
& Polynomial Equations Module\\

\midrule

\multirow{3}{0.3\textwidth}{Software Decision Module} & 
Data Structure Module\\
& Polynomial Solver Module\\
& Plotting Module \\
\bottomrule

\end{tabular}
\caption{Module Hierarchy}
\label{TblMH}
\end{table}

\section{Connection Between Requirements and Design} \label{SecConnection}

The design of the system is intended to satisfy the requirements developed in
the SRS. In this stage, the system is decomposed into modules. The connection
between requirements and modules is listed in Table \ref{TblRT}.

\section{Module Decomposition} \label{SecMD}

Modules are decomposed according to the principle of ``information hiding''
proposed by \citet{ParnasEtAl1984}. The \emph{Secrets} field in a module
decomposition is a brief statement of the design decision hidden by the
module. The \emph{Services} field specifies \emph{what} the module will do
without documenting \emph{how} to do it. For each module, a suggestion for the
implementing software is given under the \emph{Implemented By} title. If the
entry is \emph{OS}, this means that the module is provided by the operating
system or by standard programming language libraries.  Also indicate if the
module will be implemented specifically for the software.

Only the leaf modules in the
hierarchy have to be implemented. If a dash (\emph{--}) is shown, this means
that the module is not a leaf and will not have to be implemented. Whether or
not this module is implemented depends on the programming language
selected.

\subsection{Hardware Hiding Modules (\mref{mHH})}

\begin{description}
\item[Secrets:]The data structure and algorithm used to implement the virtual
  hardware.
\item[Services:]Serves as a virtual hardware used by the rest of the
  system. This module provides the interface between the hardware and the
  software. So, the system can use it to display outputs or to accept inputs.
\item[Implemented By:] OS
\end{description}

\subsubsection{Output Window Module (\mref{mOW})}

\begin{description}
\item[Secrets:]The structure on which the output data is displayed on.
\item[Services:]Creates a window in the OS that is used by the library to paint on. 
\item[Implemented By:] Python Canvas
\end{description}

\subsection{Behaviour-Hiding Module(\mref{mBH})}

\begin{description}
\item[Secrets:]The contents of the required behaviours.
\item[Services:]Includes programs that provide externally visible behaviour of
  the system as specified in the software requirements specification (SRS)
  documents. This module serves as a communication layer between the
  hardware-hiding module and the software decision module. The programs in this
  module will need to change if there are changes in the SRS.
\item[Implemented By:] --
\end{description}

\subsubsection{Exception Raising Module (\mref{mER})}

\begin{description}
\item[Secrets:] The format and structure of the input data.
\item[Services:] Checks against the input format and the known secret format and raises an exception if they do not match.
\item[Implemented By:] InputParser.py
\end{description}

\subsubsection{Output Format Module (\mref{mOF})}

\begin{description}
\item[Secrets:]What the resulting scale of the graph should be. 
\item[Services:]Determines the scale of the graph from the data.
  input parameters module.
\item[Implemented By:] Scale.py
\end{description}

\subsubsection{Polynomial Equations Module (\mref{mPE})}

\begin{description}
\item[Secrets:]A polynomial that can press through n points.  
\item[Services:]Finds the function that passes through all points in the dataset. 
  input parameters module.
\item[Implemented By:] Graph.py
\end{description}


\subsection{Software Decision Module  (\mref{mSD})}

\begin{description}
\item[Secrets:] The design decision based on mathematical theorems, physical
  facts, or programming considerations. The secrets of this module are
  \emph{not} described in the SRS.
\item[Services:] Includes data structure and algorithms used in the system that
  do not provide direct interaction with the user. 
  % Changes in these modules are more likely to be motivated by a desire to
  % improve performance than by externally imposed changes.
\item[Implemented By:] --
\end{description}

\subsubsection{Data Structure Module (\mref{mDS})}
\begin{description}
\item[Secrets:]The format and structure of the input data.
\item[Services:]Converts the input data into the data structure used by the
  input parameters module.
\item[Implemented By:] InputParser.py
\end{description}

\subsubsection{Polynomial Solver Module (\mref{mPS})}
\begin{description}
\item[Secrets:]A mathematical equation that determines a smooth curve passing through all points.
\item[Services:]Finds the function that passes through all points in the dataset. 
  input parameters module.
\item[Implemented By:] Graph.py
\end{description}

\subsubsection{Plotting Module (\mref{mP})}
\begin{description}
\item[Secrets:] X and Y direction offsets between the graphical window and the cartesian coordinate system drawn on it.
\item[Services:]Plots any (x,y) point on the cartesian coordinate system that is painted on the window. 
\item[Implemented By:] Graph.py
\end{description}

\section{Traceability Matrix} \label{SecTM}%Louis

This section shows two traceability matrices: between the modules and the
requirements and between the modules and the anticipated changes.

% the table should use mref, the requirements should be named, use something
% like fref
\begin{table}[H]
\centering
\begin{tabular}{p{0.2\textwidth} p{0.6\textwidth}}
\toprule
\textbf{Req.} & \textbf{Modules}\\
\midrule
R1 & \mref{mHH}, \mref{mInput}, \mref{mParams}, \mref{mControl}\\
R2 & \mref{mInput}, \mref{mParams}\\
R3 & \mref{mVerify}\\
R4 & \mref{mOutput}, \mref{mControl}\\
R5 & \mref{mOutput}, \mref{mODEs}, \mref{mControl}, \mref{mSeqDS}, \mref{mSolver}, \mref{mPlot}\\
R6 & \mref{mOutput}, \mref{mODEs}, \mref{mControl}, \mref{mSeqDS}, \mref{mSolver}, \mref{mPlot}\\
R7 & \mref{mOutput}, \mref{mEnergy}, \mref{mControl}, \mref{mSeqDS}, \mref{mPlot}\\
R8 & \mref{mOutput}, \mref{mEnergy}, \mref{mControl}, \mref{mSeqDS}, \mref{mPlot}\\
R9 & \mref{mVerifyOut}\\
R10 & \mref{mOutput}, \mref{mODEs}, \mref{mControl}\\
R11 & \mref{mOutput}, \mref{mODEs}, \mref{mEnergy}, \mref{mControl}\\
\bottomrule
\end{tabular}
\caption{Trace Between Requirements and Modules}
\label{TblRT}
\end{table}

\begin{table}[H]
\centering
\begin{tabular}{p{0.2\textwidth} p{0.6\textwidth}}
\toprule
\textbf{AC} & \textbf{Modules}\\
\midrule
\acref{acHardware} & \mref{mHH}\\
\acref{acInput} & \mref{mInput}\\
\acref{acGraphType} & \mref{mInput}\\
\bottomrule
\end{tabular}
\caption{Trace Between Anticipated Changes and Modules}
\label{TblACT}
\end{table}

\section{Use Hierarchy Between Modules} \label{SecUse}%Louis

In this section, the uses hierarchy between modules is
provided. \citet{Parnas1978} said of two programs A and B that A {\em uses} B if
correct execution of B may be necessary for A to complete the task described in
its specification. That is, A {\em uses} B if there exist situations in which
the correct functioning of A depends upon the availability of a correct
implementation of B.  Figure \ref{FigUH} illustrates the use relation between
the modules. It can be seen that the graph is a directed acyclic graph
(DAG). Each level of the hierarchy offers a testable and usable subset of the
system, and modules in the higher level of the hierarchy are essentially simpler
because they use modules from the lower levels.

\begin{figure}[H]
\centering
%\includegraphics[width=0.7\textwidth]{UsesHierarchy.png}
\caption{Use hierarchy among modules}
\label{FigUH}
\begin{center}
	\includegraphics[width=\linewidth]{Images/useDiag.png}
\end{center}
\end{figure}

%\section*{References}

\bibliographystyle {plainnat}
\bibliography {MG}

\end{document}